\documentclass{beamer}
\usetheme[faculty=fi]{fibeamer}

\usepackage{polyglossia}
\setmainlanguage{english}

\usepackage{minted}
\usemintedstyle{monokai}

\usepackage[hybrid,footnotes,fencedCode]{markdown}
\markdownSetup{
  renderers = {
    headingOne = {\section{#1}},
    headingTwo = {\subsection{#1}},
    headingThree = {\frametitle{#1}},
    headingFour = {\framesubtitle{#1}},
    footnote = {\footnote[frame]{#1}},
    link = {\kern-.5ex\footnote[frame]{See \url{#3}}},
    inputFencedCode = {\inputminted{\ifx\relax#2\relax tex\else #2\fi}{#1}},
  }
}

\title{A \raisebox{-1.7mm}{\includegraphics[scale=0.2]{markdown-mark}}\ Interpreter for \TeX}
\subtitle{Project MUNI 33 / 12 2015}
\author{Vít Novotný}
\frenchspacing
\begin{document}
\frame{\maketitle}

\AtBeginSection[]{% Print an outline at the beginning of sections
  \begin{frame}<beamer>
    \frametitle{Outline for Section \thesection}
    \tableofcontents[currentsection]
  \end{frame}}

\begin{darkframes}
\begin{frame}{Table of Contents}
  \tableofcontents
\end{frame}
\begin{markdown}
# Introduction
## The Case for Lightweight Markup
\begin{frame}
### \subsecname
#### What is Wrong with \TeX?

  1. High Markup to Text Ratio
    * The \TeX book <http://mirrors.ctan.org/systems/knuth/dist/tex/texbook.tex> by Donald E. Knuth is 22\,\% markup.
    * Think Java <http://github.com/AllenDowney/ThinkJava> by Allen Downey and Chris Mayfield is 21\,\% markup.
    %* Efficient R<https://github.com/csgillespie/efficientR/> by Colin Gillespie and Robin Lovelace is 8\,\% markup
  2. Zero Sandboxing Support
    * The document you are typesetting may not compile.
    ```
    … the file is named \texttt{evil_underscores.tex} …
    ```
    * The document you are typesetting may access the system shell.
    ```
    \immediate\write18{rm -rf /} % Don't try this at home.
    ```
  3. Moderately Steep Learning Curve

\end{frame}
## Existing Solutions
# The Markdown Package
## Context-Sensitive Parsing
## Lua\TeX{} and Lunamark
## User Interface
# Conclusion
\end{markdown}
\end{darkframes}
\end{document}
