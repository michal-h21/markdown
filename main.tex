\documentclass{beamer}
\usetheme[faculty=fi]{fibeamer}

% Localization
\usepackage{polyglossia}
\setmainlanguage{english}

% Markdown setup
\usepackage[
  hybrid,
  footnotes,
  fencedCode,
  citations,
]{markdown}
\markdownSetup{
  renderers = {
    emphasis = {\alert{#1}},
    headingOne = {\section{#1}},
    headingTwo = {\subsection{#1}},
    headingThree = {\frametitle{#1}},
    headingFour = {\framesubtitle{#1}},
    footnote = {\footnote[frame]{#1}},
    horizontalRule = {\framebreak},
  }
}

% Bibliography setup
\usepackage{filecontents}
\begin{filecontents}{main.bib}

@BOOK{knuth86,
  author        =   {Knuth, Donald Ervin},
  year          =   {1986},
  title         =   {The \TeX book},
  edition       =   {3},
  isbn          =   {0-201-13447-0},
  pagetotal     =   {ix, 479},
  publisher     =   {Addison-Westley},
  langid        =   {english},
  url           =   {https://mirrors.ctan.org/systems/knuth/dist/tex/texbook.tex},
  urldate       =   {2016-11-08},
}

@ONLINE{gruber13,
  author        =   {John Gruber},
  year          =   {2013},
  title         =   {Markdown},
  urldate       =   {2016-08-15},
  url           =   {https://daringfireball.net/projects/markdown/}, 
  langid        =   {english},
}

@ARTICLE{dominici14,
  author        =   {Massimiliano Dominici},
  year          =   {2014},
  title         =   {An overview of Pandoc},
  journaltitle  =   {TUGboat},
  volume        =   {35},
  number        =   {1},
  urldate       =   {2016-08-15},
  url           =   {http://tug.org/TUGboat/tb35-1/tb109dominici.pdf},
  pages         =   {44--50},
  issn          =   {0896-3207},
  langid        =   {english},
}

@ONLINE{macfarlane16,
  author        =   {John MacFarlane},
  year          =   {2016},
  title         =   {Pandoc},
  subtitle      =   {a universal document converter},
  urldate       =   {2016-08-15},
  url           =   {http://pandoc.org/}, 
  langid        =   {english}}

@BOOK{downey16,
  author        =   {Downey, Allen B. and Mayfield, Chris},
  year          =   {2016},
  title         =   {Think Java},
  subtitle      =   {How to Think Like a Computer Scientist},
  version       =   {6.1.0},
  pagetotal     =   {xviii, 273},
  publisher     =   {Green Tea Press},
  langid        =   {english},
  url           =   {https://thinkjava.org/},
  urldate       =   {2016-11-08},
}

@BOOK{gillespie16,
  author        =   {Gillespie, Colin and Lovelace, Robin},
  year          =   {2016},
  title         =   {Efficient R programming},
  isbn          =   {978-1-4919-5078-4},
  pagetotal     =   {204},
  publisher     =   {O'Reilly Media},
  langid        =   {english},
  url           =   {https://github.com/hadley/r4ds/},
  urldate       =   {2016-11-08},
}

@BOOK{grolemund16,
  author        =   {Grolemund, Garrett and Wickham, Hadley},
  year          =   {2016},
  title         =   {R for Data Science},
  isbn          =   {978-1-4919-1039-9},
  pagetotal     =   {518},
  publisher     =   {O'Reilly Media},
  langid        =   {english},
  url           =   {https://github.com/hadley/r4ds/},
  urldate       =   {2016-11-08},
}

\end{filecontents}
\usepackage[
  backend=biber,
  style=iso-authoryear,
  sorting=nty,
  autolang=other,
  sortlocale=auto,
]{biblatex}
\addbibresource{main.bib}

% Miscellaneous packages and other setup
\usepackage{minted}
\usemintedstyle{monokai}
\frenchspacing

% Metadata
\title{A \raisebox{-1.7mm}{\includegraphics[scale=0.2]{markdown-mark}}\ Interpreter for \TeX}
\subtitle{Project MUNI 33 / 12 2015}
\author{Vít Novotný}
\begin{document}
\frame{\maketitle}

\AtBeginSection[]{% Print an outline at the beginning of sections
  \begin{frame}<beamer>
    \frametitle{Outline for Section \thesection}
    \tableofcontents[currentsection]
  \end{frame}}

\begin{darkframes}
\begin{frame}{Table of Contents}
  \tableofcontents
\end{frame}
\begin{markdown}
# Introduction
## The Case for Lightweight Markup

\begin{frame}

### \subsecname
#### What is Wrong with \TeX?

  1. High Markup to Text Ratio
    * @knuth86 is 22\,\% markup.
    * @downey16 is 21\,\% markup.
  2. Zero Sandboxing Support
    * The document you are typesetting may not compile.
    ```tex
    … a file named \texttt{evil_underscores.tex} …
    ```
    * The document you are typesetting may halt.
    ```tex
    \newcommand\whiletrue{\whiletrue} \whiletrue
    ```
    * The document you are typesetting may access the system shell.
    ```tex
    \immediate\write18{sudo rm -rf /}
    ```
  3. Steep Learning Curve

\end{frame}
\begin{frame}

### \subsecname
#### The Language of Markdown

> The overriding design goal for Markdown’s formatting syntax is to make it
> *as readable as possible*. The idea is that a Markdown-formatted document
> should be _publishable as-is, as plain text_, without looking like it’s been
> marked up with tags or formatting instructions. While Markdown’s syntax has
> been influenced by several existing text-to-html filters, the single
> biggest source of inspiration for Markdown’s syntax is *the format of plain
> text email*.

\hfill --- @gruber13, emphasis mine

\end{frame}
\begin{frame}

### \subsecname
#### The Language of Markdown

```markdown
# This is a level one title #
This is a text paragraph with _emphasis_.
> This paragraph will show as a quote.

␣␣␣␣This is is a source code example.

* First item with **strong emphasis**
* Second item with a [link](http://link.com/ "Title")

1. First item with `inline code`.
2. Second item with an ![image](image.png "Title")
```

\end{frame}
\begin{frame}

### \subsecname
#### How is Markdown Useful?

  1. Minimal Markup to Text Ratio
    * Recall: @knuth86 and @downey16 are _\textasciitilde 22\,\% markup_.
    * @gillespie16 is 5.5\,\% markup.
    * @grolemund16 is 3.8\,\% markup.
  2. Sandboxing Support
    * A Markdown document converted to \TeX{} will always compile.
    * The document may neither halt nor access the shell.
  3. Hybrid Markup Support
    * Markdown was designed to supplement html, not replace it.
    * Structurally simple sections can use pure Markdown, complex sections
      may combine Markdown and the host markup.
  4. Mild Learning Curve

\end{frame}

## Existing Solutions
# The Markdown Package
## Context-Sensitive Parsing
## Lua\TeX{} and Lunamark
## User Interface
# Conclusion
# Bibliography

\begin{frame}[allowframebreaks]
### \secname
\printbibliography
\end{frame}

\end{markdown}
\end{darkframes}
\end{document}
